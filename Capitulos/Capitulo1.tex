%--------------------------------------------------------------------------------------
%	Capítulo 1
%---------------------------------------------------------------------------------------
\chapter{Capítulo 1: Contenido y Estilos en Latex}
Este capítulo sirve como modelo, es decir, para mostrar cómo utilizar latex. También muestra el posible contenido del libro. Un ejemplo de referencia bibliográfica es: \cite{kappel2006web}.


\begin{caja}[Advertencia.]
	Las siguientes plantillas usan la versión 2014 del paquete
	\verb+tcolorbox+ (entre otros paquetes recientes),  por lo tanto {\it debe actualizar los paquetes de sus distribución} \TeX{} o instalar manualmente este paquete (ver el capítulo 9 del libro, \url{http://www.tec-digital.itcr.ac.cr/revistamatematica/Libros/LATEX/LaTeX_2014.pdf}). El paquete "psboxit" viene incluido en la carpeta.
\end{caja}



%---------ESTRUCTURA GENERAL DEL LIBRO-------------_-
%------------------------------------------
\section{Estructura General del Libro}

\begin{enumerate}
	\item \textsc{Análisis de Algoritmos}
	\begin{enumerate}
		\item Los algoritmos
		\item El análisis de algoritmos
		\item Función de complejidad
		\item Cómo calcular la funcion de complejidad de un algoritmo
		\item Orden de Magnitud (Notación O Grande)
		\item Complejidad de un Algoritmo Recursivo
		\item Ejercicios Propuestos
	\end{enumerate}
	
	\item \textsc{Introduccion a las estructuras de datos}
	\begin{enumerate}
		\item Conceptos básicos sobre estructuras de datos.
		\item Clasificación.
		\begin{enumerate}
			\item Estructuras de Datos Estáticas.
			\item Estructuras de Datos Dinámicas.
		\end{enumerate}
	\end{enumerate}
	
	\item \textsc{Tipos abstractos de datos - TAD}
	\begin{enumerate}
		\item Tipos de datos
		\item Tipos abstractos de datos
		\item Métodos para la Especificación de un TAD
		\item Tipos de operaciones
		\item Ejemplos de TADs
		\item Ejercicios Propuestos
	\end{enumerate}
	
	\item \textsc{LISTAS DINÁMICAS}
	\begin{enumerate}
		\item Definición
		\item Usos de las listas
		\item El TAD Lista
		\item Implementación del TAD Lista orientado a objetos
		\item Implementación del TAD lista mediante arreglos
		\item Casos de estudio
		\item Ejercicios propuestos
	\end{enumerate}
	
	\item \textsc{PILAS}
	\begin{enumerate}
		\item Definición
		\item Usos de las pilas
		\item El TAD PIla
		\item Implementación del TAD pila orientado a objetos
		\item Implementación del TAD pila mediante arreglos
		\item Casos de estudio
		\begin{enumerate}
			\item Correspondencia de delimitadores
			\item Evaluación de expresiones aritméticas
			\item Convertir una expresión dada en notación infija a una notación postfija 
			\item Evaluación de la Expresión en notación postfija
		\end{enumerate}		
		\item Ejercicios propuestos
	\end{enumerate}
		
	\item \textsc{COLAS}
	\begin{enumerate}
		\item Definición
		\item Usos de las colas
		\item El TAD Cola
		\item Implementación del TAD cola orientado a objetos
		\item Implementación del TAD cola mediante arreglos
		\item Casos de estudio
		\begin{enumerate}
			\item Correspondencia de delimitadores
			\item Evaluación de expresiones aritméticas
			\item Convertir una expresión dada en notación infija a una notación postfija 
			\item Evaluación de la Expresión en notación postfija
		\end{enumerate}		
		\item Ejercicios propuestos
	\end{enumerate}		

	\item \textsc{Estructuras de Datos No Lineales. Arboles Binarios}
	\begin{enumerate}
		\item Introducción
		\item Definición de árbol
		\item Definición de árbol binario
		\item Árbol de expresiones
		\item Balance o equilibrio de un árbol binario
		\item Árbol binario completo
		\item TAD Árbol binario
		\item Implementación del TAD de un árbol binario
		\item Recorridos de un árbol
		\begin{enumerate}
			\item Recorrido inorden
			\item Recorrido en preorden
			\item Recorrido en postorden
			\item Recorrido en anchura
		\end{enumerate}		
		\item Árbol binario de búsqueda
		\begin{enumerate}
			\item Operación de inserción
			\item Operación de búsqueda
			\item Operación de eliminación
		\end{enumerate}	
		\item Árbol binario de búsqueda equilibrados AVL
		\begin{enumerate}
			\item Eficiencia en la búsqueda de un árbol equilibrado
			\item Inserción en árboles AVL
			\item Borrado de un nodo en un árbol AVL
		\end{enumerate}	
		\item Ejercicios propuestos
	\end{enumerate}	
			
	\item \textsc{Estructuras de Datos No Lineales. Arboles N-ARIOS}
	\begin{enumerate}
		\item Introducción
		\item Definiciones y conceptos básicos
		\item El TAD ArbolN
		\item Implementación del TAD ArbolN
		\item Ejercicios propuestos
	\end{enumerate}	
					
	\item \textsc{ARBOL1-2-3: UN ÁRBOL TRIARIO ORDENADO}
	\begin{enumerate}
		\item Introducción
		\item Definiciones
		\item El TAD ARBOL1-2-3
		\item Implementación del TAD ARBOL1-2-3
		\item Ejercicios propuestos
	\end{enumerate}			
	
	\item \textsc{ARBOL2-3: UN ÁRBOL TRIARIO ORDENADO}
	\begin{enumerate}
		\item Introducción
		\item Definiciones
		\item Un árbol B 
		\item El TAD ARBOL2-3
		\item Implementación del TAD ARBOL2-3
		\item Ejercicios propuestos
	\end{enumerate}
	
	\item \textsc{TRIE: CONJUNTO DE PALABRAS}
	\begin{enumerate}
		\item Introducción
		\item Definiciones
		\item El TAD TRIE
		\item Implementación del TAD TRIE
		\item Ejercicios propuestos
	\end{enumerate}	
	
	\item \textsc{CUADTREE: REPRESENTACIÓN DE IMÁGENES}
	\begin{enumerate}
		\item Introducción
		\item Definiciones
		\item El TAD CUADTREE
		\item Implementación del TAD CUADTREE
		\item Ejercicios propuestos
	\end{enumerate}	
	
	\item \textsc{Estructura dinámicas no lineales: Grafos}
	\begin{enumerate}
		\item Introducción
		\item Definiciones
		\item El TAD Grafo
		\item Representación de los Grafos
		\begin{enumerate}
			\item Matriz de adyacencia
			\item Implementación de la Matriz de Adyacencia
			\item Listas de adyacencia
			\item Implementación de la lista de Adyacencia
		\end{enumerate}	
		\item Recorridos de un Grafo
		\begin{enumerate}
			\item Recorrido en anchura
			\item Recorrido en profundidad
		\end{enumerate}			
		\item Conexiones en un grafo
		\begin{enumerate}
			\item Componentes conexas de un grafo
			\item Matriz de caminos, cierre transitivo
			\item Matriz de caminos y cierre transitivo
		\end{enumerate}	
		\item Matriz de caminos:Algoritmo de Warshall
		\item Algoritmo de costos mínimos:Dijkstra
		\item Algoritmo de Floyd
		\item Ejercicios propuestos
	\end{enumerate}							
								
\end{enumerate}		

\section{Prueba de entornos}

\begin{definicion}[(Igualdad)][cap1:Igualdad]
  $$a=b$$
\end{definicion}

\bigskip
Según la definición \ref{cap1:Igualdad}, la igualdad...\\

\bigskip
\begin{teorema}
 $$a=b$$
\end{teorema}

\bigskip
\begin{ejemplo}
 $$a=b$$
\end{ejemplo}

%------------------------------------------------------------------------------------------
\clearpage

\begin{lema}
  $$a=b$$
\end{lema}

\bigskip
\begin{corolario}
 $$a=b$$
\end{corolario}

\bigskip
\begin{caja}[Una caja de comentario]
 $$a=b$$
\end{caja}

\subsection{Tablas}
 % \usepackage{array,tabularx}
 % \usepackage[table, x11names]{xcolor}

\newcolumntype{Y}{>{\raggedleft\arraybackslash}X} % Ver tabularx
%
\tcbset{enhanced,fonttitle=\bfseries\large,fontupper=\normalsize\sffamily,
         colback=LightCyan1,colframe=DarkOrange4,colbacktitle=DarkOrange4,
         coltitle=black,center title
       }

%%-Tabla - Estilo beamer
\begin{tcolorbox}[tabularx={X||X||X},title= {\white Iteración}, beamer]
  & $x_i$           & $y_i=f(x_i)$ \\\hline\hline
A & $x_0=0$         & $0$           \\\hline
B & $x_1=0.75$      &  $-0.0409838$ \\\hline
C & $x_2=1.5$       &  $1.31799$  
\end{tcolorbox}


%EJEMPLO CODIGO FUENTE EN JAVA

\subsection{Ejemplo codigo fuente Java}

\begin{lstlisting}[language=Java]
package com.unicauca.ejemplo;
public class Hello {
//Comentario
/*Comentario*/
/**Comentario*/
public static void main(String[] args) {
System.out.println("Hola mundo");
}
}
\end{lstlisting}
